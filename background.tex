As indicated in Figure~\ref{fig:sigRegionBG}, different Standard Model processes contribute to the event sample in the signal region. To distinguish a potential signal from these backgrounds, a precise estimation of the background contributions is mandatory. While the simulation of these processes and the response of the CMS detector gives a good description of the data for the majority of the phase space, a large number of uncertainty sources are introduced in the modelling of the physical process and the detector. Therefore a higher precision can be achieved by deriving the background estimates directly from the recorded data. The background processes are categorized as either being flavour-symmetric or as containing the production of a Z boson. A dedicated method is applied for each of the two categories. 
\section{Flavour-symmetric backgrounds}
Processes that are symmetric in the production of same-flavour and opposite-flavour lepton pairs allow for the estimation of their contribution to the SF event sample from the OF one. The most dominant of these processes is the dileptonic decay of top-pair production, where the leptons are produced uncorrelatedly in the decay of the W bosons. Other examples are the decays of two $\tau$ leptons, which are in turn produced in the decay of a Z boson or the dileptonic decay of W pairs. Another contribution to this class of backgrounds are misidentified leptons, as will be demonstrated later. 

No significant deviation from flavour-symmetry has been observed in the decays of the W boson, with a measured ratio of the branching fractions into $e+\nu$ and $\mu + \nu$ of $1.007\pm0.021$. In the decays of the $\tau$ lepton the different masses of electron and muon have a noticeable effect, resulting in a slightly favoured decay into electrons. Here the ratio of branching fractions is $1.0241\pm0.0032$~\cite{PDG}. As backgrounds with $\tau$ leptons are a sub-dominant contribution to the the flavour-symmetric backgrounds, these can be considered to be fully flavour-symmetric on particle level. However, distortions of the flavour-symmetry are introduced by the different efficiencies for triggering, reconstructing, and identifying electrons and muons in CMS. The background estimation from OF events therefore has to include a correction for this deviation, which is applied as a multiplicative factor:
\begin{equation}
N_{SF}^{pred} = \Rsfof \cdot N_{OF}.
\end{equation}
Two independent methods are utilized to measure \Rsfof on data. In the first approach is is directly measured as the ratio of SF to OF events in the control region for flavour-symmetric backgrounds. The second approach studies the lepton efficiencies and derives \Rsfof factorized into the effects of trigger efficiencies and reconstruction and identification efficiencies.  

\subsection{Direct measurement of \Rsfof}

\subsection{Determination of \Rsfof with the factorization method}
Asymmetries between the lepton flavours introduced by differing reconstruction and selection efficiencies can be corrected for if the ratio of efficiencies for muons and electrons $\rmue = \frac{\epsilon_{\mu}}{\epsilon_{e}}$ is known. Under the assumption that the efficiencies for the two leptons in the event factorize, i.e. $\epsilon_{ll} = \epsilon_{l}\cdot\epsilon_{l}$, the number of dielectron and dimuon events can be estimated from the opposite-flavour events using the relations
\begin{equation}
n_{ee}^{*} = \frac{1}{2}\cdot \frac{n_{OF}^{*}}{\rmuestar}
\end{equation}
and 
\begin{equation}
n_{\mu\mu}^{*} = \frac{1}{2} \cdot \rmuestar \cdot n_{OF}^{*},
\end{equation}
the $^{*}$ indicating that these are the values unaffected by trigger efficiencies.
The prediction of the combined same-flavour yield is therefore given by
\begin{equation}
n_{SF}^{*} = \frac{1}{2}\cdot \left( \rmuestar + \frac{1}{\rmuestar} \right) n_{OF}^{*}.
\end{equation}
In practice, all measured quantities are affected by the efficiencies of the different dilepton triggers. The measured number of SF events will therefore be
\begin{equation}
n_{SF} = \effeet \cdot n_{ee}^{*} + \effmmt \cdot n_{\mu\mu}^{*},
\end{equation}
where $\epsilon_{ll}^T$ denotes the trigger efficiency for the given dilepton combination.\\
Also the predictions for $n_{ee}^{*}$ and $n_{\mu\mu}^{*}$ have to include the trigger efficiencies. Here \rmuestar is expressed in terms of the measured value \rmue, which is derived from the \EE and \MM event yields in the Drell-Yan control region (see Section~\ref{sec:rmue}) as 
\begin{equation}
\rmue  = \sqrt{\frac{N_{\mu\mu}}{N_{ee}}} \approx \sqrt{\frac{\epsilon_{\mu}^2 \effmmt}{\epsilon_{e}^2 \effeet}} = \rmuestar \cdot \sqrt{\frac{\effmmt}{\effeet}}.
\end{equation}
Taking into account also that the measured event yield in the OF channel is $n_{OF} = \effemt \cdot n_{OF}^{*}$, the estimate for the yields in the same-flavour channel becomes
\begin{equation}
n_{ee} = \frac{1}{2\rmue} \cdot \frac{\sqrt{\effeet \effmmt}}{\effemt} n_{OF}
\end{equation} 
and
\begin{equation}
n_{\mu\mu} = \frac{1}{2}\rmue  \cdot \frac{\sqrt{\effeet \effmmt}}{\effemt} n_{OF}.
\end{equation} 
Finally, the combined prediction of the SF yield is
\begin{equation}
n_{SF} = \frac{1}{2}\left(\rmue + \frac{1}{\rmue}\right) \cdot \frac{\sqrt{\effeet \effmmt}}{\effemt}  n_{OF}.
\end{equation}
\subsubsection{Measurement of \rmue}
\label{sec:rmue}
\subsubsection{Measurement of \RT}

\section{Backgrounds containing a Z boson}
\section{Investigation of possible further backgrounds}
\section{Search for a kinematic edge with a fit}
