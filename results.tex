\label{sec:counting}
In the counting experiment approach, the observed yield of SF events is compared to the combined background estimates from flavour-symmetric and Drell--Yan backgrounds in the six regions defined in \mll and lepton $|\eta|$. Here, the results are presented their basic properties are discussed.
\section{Results and further studies}

\label{sec:candcresults}
The distribution of the dilepton invariant mass in the central and forward signal regions (see Table~\ref{tab:selections}) are shown in Figure~\ref{fig:resultsCC}. The resulting event yields are compared to the expectation from the backgrounds in Table~\ref{tab:METresults2012}. A maximum likelihood estimator for the difference of expected and observed yield is determined in each region. For this purpose, a likelihood function is defined~\cite{HiggsTool1}
\begin{equation}
\label{eq:ML}
\mathcal{L}(\text{data}|\mu,\theta) = \text{Poisson}\left(\text{data}| s\left(\theta\right) + b\left(\theta\right)\right)\cdot p\left(\tilde{\theta}|\theta\right),
\end{equation}
where $s$ and $b$ represent the number of signal and background events, and $p(\tilde{\theta}|\theta)$ parametrises the uncertainties, also known as nuisance parameters, $\theta$, with $\tilde{\theta}$ being the nominal value of these parameters. This likelihood is fit to the observed data by minimizing its negative logarithm. The resulting value of the number of signal events $s$ is used as the estimator for this quantity. The significances of deviations of this difference from zero are evaluated using a profile likelihood ratio of the signal and signal plus background hypotheses~\cite{HiggsTool1}. In general, the observed data is in agreement with the background estimation within about one standard deviation, except for the low-mass region for central leptons. Here, the observed yield exceeds the expectation by $109\pm48$ events. The size of this excess corresponds to a local significance of 2.2~$\sigma$.  
\begin{figure}[htbp]
\centering
\begin{minipage}[t]{0.49\textwidth}
  \includegraphics[width=\textwidth]{plots/results/mllResult_SignalCentral_Full2012_SF.pdf}
\end{minipage}
\begin{minipage}[t]{0.49\textwidth}
\includegraphics[width=\textwidth]{plots/results/mllResult_SignalForward_Full2012_SF.pdf}
\end{minipage}

\caption{Distribution of \mll in the signal region for the central (left) and forward (right) dilepton selection. The data is shown as black dots, while the total background prediction from data is shown as a blue histogram. The blue error bars indicate the combined statistical and systematic background uncertainty in each bin. The contribution from Drell--Yan backgrounds is shown as a green histogram. The dashed lines indicate the boundaries of the three mass bins. Beneath the plot the ratio of data to the background prediction is shown. The error bars include the statistical uncertainties of data and background, while the blue band indicates the systematic uncertainties on the background. }
\label{fig:resultsCC}
\end{figure} 


\begin{table}[btp]
 \renewcommand{\arraystretch}{1.3}
 \setlength{\belowcaptionskip}{6pt}
 \scriptsize
 \centering
 \caption{Results of the counting experiment in the six signal regions.
     The statistical and systematic uncertainties are added in quadrature, except for the flavor-symmetric backgrounds. The presented differences between the observed and estimated yields are obtained with a maximum likelihood fit (see text).    Low-mass refers to $20\GeV < \mll < 70$\GeV, on-\Z to  $81\GeV < \mll < 101$\GeV, and high-mass to $\mll > 120$\GeV.
     }
  \label{tab:METresults2012}
  \begin{tabular}{l| cc | cc | cc}

    							& \multicolumn{2}{c}{low-mass} & \multicolumn{2}{c}{on-\Z} & \multicolumn{2}{c}{high-mass} \\ 

    \hline
                                &  Central        & Forward  &  Central  & Forward   &  Central        & Forward \\ 

    \hline
        Observed       &  865                   & 154              &  494            &  176       &   849           &   381    \\

    \hline
        Flav.-sym.    & $746\pm27\pm26$        & $144\pm12\pm7$  &  $368\pm19\pm13$ & $137\pm11\pm7$ & $789\pm28\pm28$ & $411\pm20\pm21$ \\

            Drell--Yan          & $8.6\pm2.7$            & $2.6\pm0.8$      & $119\pm21$ & $43\pm9$ & $2.7\pm0.8$ & $1.2\pm0.4$ \\

    \hline
            Total est.          & $755\pm38$            & $147\pm14$      & $488\pm31$ & $180\pm16$ & $792\pm39$ & $413\pm30$ \\

    \hline
         Obs. - est.  & $109\pm48$      & $7\pm19$ & $6\pm38 $ & $-5\pm21$ & $57\pm50$ & $-32\pm37 $ \\ 

    \hline
   Significance      & 2.2~$\sigma$    &  0.4~$\sigma$  & 0.1~$\sigma$ & $<$0.1~$\sigma$ & 1.1~$\sigma$ & $<$0.1~$\sigma$ \\ 


  \end{tabular}
\end{table}



In Tables~\ref{tab:METresults2012EE} and~\ref{tab:METresults2012MM}, the results are shown separately for \EE and \MM events. As expected since \rmue is larger than one, the yields in the \MM channel are slightly larger than those in the \EE channel. For the flavour-symmetric backgrounds, and therefore also for the total background estimates and the difference of observation and estimation, the yields in the \EE and \MM channels do not exactly add up to those in the combined SF channel presented in Table~\ref{tab:METresults2012}. This is caused by the weighted average of the two methods to determine the correction factors used to translate from OF into the different SF channels, which is calculated separately for \Rsfof, \Reeof, and \Rmmof, as discussed in Section~\ref{sec:combinedRSFOF}. Comparing the two channels, consistent results are observed between them, except for the slight excess in the high-mass central region, which is dominated by \EE events. Especially for the larger excess in the low-mass central region, the observation agrees with the expected behaviour of the different flavours. The signal yields of $47\pm25$ and $67\pm29$ in the \EE and \MM channels correspond to a value of \rmue for this hypothetical signal of $1.19\pm0.41$, in good agreement with the value of $1.09\pm0.11$ measured in the Drell--Yan control region (see Section~\ref{sec:rmue}). However, to explain the observed excess with a value of \Rsfof = 1.17 with a change in the lepton efficiencies, an \rmue of about 1.7 is necessary, as illustrated in Figure~\ref{fig:rmuePropaganda}. A further discussion of these results can be found in Chapter~\ref{sec:newInt}.

\begin{table}[hbtp]
 \renewcommand{\arraystretch}{1.3}
 \setlength{\belowcaptionskip}{6pt}
 \scriptsize
 \centering
 \caption{Results of the counting experiment for \EE events only.
     The statistical and systematic uncertainties are added in quadrature, except for the flavor-symmetric backgrounds.
     Low-mass refers to $20 < \mll < 70$\GeV, on-\Z to  $81 < \mll < 101$\GeV and high-mass to $\mll > 120$\GeV.
     }
  \label{tab:METresults2012EE}
  \begin{tabular}{l| cc | cc | cc}

    							& \multicolumn{2}{c}{low-mass} & \multicolumn{2}{c}{on-\Z} & \multicolumn{2}{c}{high-mass} \\ 

    \hline
                                &  Central        & Forward  &  Central  & Forward   &  Central        & Forward \\ 

    \hline
        Observed       &  389                   & 53              &  232            &  86       &   401           &   195    \\

    \hline
        Flavor-symmetric    & $337\pm12\pm19$        & $61\pm5\pm6$  &  $166\pm8\pm9$ & $58\pm5\pm5$ & $357\pm12\pm21$ & $175\pm8\pm17$ \\

            Drell--Yan          & $4.3\pm1.3$            & $1.2\pm0.4$      & $62\pm11$ & $21\pm5$ & $1.5\pm0.5$ & $0.7\pm0.2$ \\

    \hline
            Total estimated          & $342\pm23$            & $62\pm8$      & $229\pm17$ & $79\pm9$ & $358\pm24$ & $175\pm19$ \\

    \hline
         Observed - estimated  & $47^{+25}_{-25}$      & $-10^{+9}_{-9}$ & $3^{+21}_{-21} $ & $6^{+12}_{-12}$ & $42^{+25}_{-26}$ & $19^{+18}_{-18} $ \\ 

    \hline
   Significance      & 1.9~$\sigma$    &  $<$0.1~$\sigma$  & 0.1~$\sigma$ & 0.5~$\sigma$ & 1.7~$\sigma$ & 1.1~$\sigma$ \\ 


  \end{tabular}
\end{table}





\begin{table}[btp]
 \renewcommand{\arraystretch}{1.3}
 \setlength{\belowcaptionskip}{6pt}
 \scriptsize
 \centering
 \caption{Results of the counting experiment for \MM events only.
     The statistical and systematic uncertainties are added in quadrature, except for the flavor-symmetric backgrounds. The presented differences between the observed and estimated yields are obtained with a maximum likelihood fit (see text).    Low-mass refers to $20\GeV < \mll < 70$\GeV, on-\Z to  $81\GeV < \mll < 101$\GeV, and high-mass to $\mll > 120$\GeV.
     }
  \label{tab:METresults2012MM}
  \begin{tabular}{l| cc | cc | cc}

    							& \multicolumn{2}{c}{low-mass} & \multicolumn{2}{c}{on-\Z} & \multicolumn{2}{c}{high-mass} \\ 

    \hline
                                &  Central        & Forward  &  Central  & Forward   &  Central        & Forward \\ 

    \hline
        Observed       &  476                   & 101              &  262            &  90       &   448           &   186    \\

    \hline
        Flav.-sym.    & $405\pm14\pm21$        & $79\pm6\pm6$  &  $200\pm10\pm10$ & $74\pm6\pm6$ & $428\pm15\pm23$ & $224\pm11\pm19$ \\

            Drell--Yan          & $4.4\pm1.4$            & $1.6\pm0.6$      & $58\pm10$ & $25\pm6$ & $1.2\pm0.4$ & $0.7\pm0.2$ \\

    \hline
            Total est.          & $409\pm26$            & $80\pm9$      & $258\pm18$ & $100\pm11$ & $429\pm27$ & $225\pm22$ \\

    \hline
         Obs. - est.  & $67\pm29$      & $20\pm13$ & $3\pm23 $ & $-11\pm13$ & $19\pm29$ & $-40\pm21 $ \\ 

    \hline
   Significance      & 2.3~$\sigma$    &  1.6~$\sigma$  & 0.2~$\sigma$ & $<$0.1~$\sigma$ & 0.6~$\sigma$ & $<$0.1~$\sigma$ \\ 


  \end{tabular}
\end{table}




