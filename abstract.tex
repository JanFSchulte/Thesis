In this thesis, a search for Supersymmetry in events with two opposite-sign same-flavour leptons, jets and missing transverse energy is presented. The considered dataset of proton-proton collisions at a centre-of-mass energy of 8$\,\mathrm{TeV}$, recorded with the CMS detector, corresponds to an integrated luminosity of 19.5\,$\mathrm{fb}^{-1}$. The analysis focusses on the correlated production of electron or muon pairs with opposite sign in the cascade decays of heavy super-symmetric particles. In the decay of a heavy neutralino into two leptons and a lighter neutralino, the mass difference between the two neutralinos sets an upper limit on the invariant mass $m_{\ell\ell}$ of the dilepton system, resulting in a characteristic edge in the $m_{\ell\ell}$ distribution. Other parts of the signal signature include several hadronic jets and, as the lightest supersymmetric particle, in this case the lighter neutralino, is assumed to be stable and leaves the detector undetected, missing transverse energy. Therefore, events are selected with at least two opposite-sign same-flavour leptons and requirements on the number of jets and the missing transverse energy. Backgrounds from known Standard Model processes contributing to this selection are categorized as either flavour-symmetric or containing the correlated production of leptons, for example in the decay of a $\mathrm{Z}$ boson. The dominant flavour-symmetric background can be estimated with high precision from events with opposite-flavour lepton pairs. Small corrections to the flavour-symmetry of these processes, caused by experimental effects, are derived on data with two independent methods, which give consistent results. The backgrounds containing flavour-correlated production of leptons are a sub-dominant contribution and also estimated from data. Deviations of the observed data from the background estimation are assessed in two ways. In a counting experiment, the number of observed events is compared to the estimated yield. In search for the characteristic edge signature, a fit to the \mll distribution is performed, using separate models for the two types of background and a triangular signal shape. The fit results in an edge position of $82.4^{+2.1}_{-3.3}$\,GeV and a signal yield of $140\pm44$ events for leptons reconstructed in the central part of the CMS detector. This results corresponds to a local significance of 2.5\,$\sigma$, which reduces to 1.7$\,\sigma$ when taking into account the fact that a signal could occur anywhere in the $m_{\ell\ell}$ distribution. The counting experiment gives consistent results. To study the impact of the result on Supersymmetry, the results of the counting experiment are interpreted in two simplified models of pair production of bottom squarks.   