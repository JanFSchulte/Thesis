%%%%%%%%%%%%     Einstellungen      %%%%%%%%%%%%%%%%%%%%%%%%%%%%%%%%%%%%%%%%%%%%%%%%%%%%
%%%%%%%%%%%%%%%%%%%%%%%%%%%%%%%%%%%%%%%%%%%%%%%%%%%%%%%%%%%%%%%%%%%%%%%%%%%%%%%%%%%%%%%%

% Grundlegende Einstellungen
\documentclass[fontsize=11pt,openright,paper=a4,pointlessnumbers,bibtotoc]{scrbook}
\usepackage[bottom=3.0cm, left=3.0cm, right=3.0cm]{geometry} 
\usepackage[squaren, thinspace ,thinqspace,binary]{SIunits}
\usepackage[latin2]{inputenc}
\usepackage[T1]{fontenc}
\newcommand{\changefont}[3]{
\fontfamily{#1} \fontseries{#2} \fontshape{#3} \selectfont}
%\changefont{pbk}{b}{sc}
\usepackage{lmodern}
\renewcommand*\familydefault{\sfdefault} %% Only if the base font of the document is to be sans serif
\usepackage[T1]{fontenc}
%\changefont{phv}{m}{n}
\usepackage{amsmath}
%\usepackage{amsfonts}
\usepackage{amssymb}
\usepackage[english]{babel} % neue deutsche Rechtschreibung
\usepackage{graphicx}
%\usepackage{picins}
\usepackage{wasysym}
\usepackage{subfigure}
\usepackage{afterpage}
\usepackage{color}
\usepackage{float}
\usepackage{mathrsfs}
\usepackage{tabularx}
\usepackage{slashed}
\usepackage{xspace}
\usepackage{rotating}
\usepackage{multirow}
%\usepackage{ptdr-definitions}
\usepackage{footnote}
\makesavenoteenv{tabular}
\makesavenoteenv{table}

\usepackage{mathtools}








\DeclarePairedDelimiter\abs{\lvert}{\rvert}%
\DeclarePairedDelimiter\norm{\lVert}{\rVert}%

% Swap the definition of \abs* and \norm*, so that \abs
% and \norm resizes the size of the brackets, and the 
% starred version does not.
\makeatletter
\let\oldabs\abs
\def\abs{\@ifstar{\oldabs}{\oldabs*}}
%
\let\oldnorm\norm
\def\norm{\@ifstar{\oldnorm}{\oldnorm*}}
\makeatother

\usepackage[small]{caption}
\usepackage[plainheadsepline,automark]{scrpage2}
\usepackage[Q=yes]{examplep}

\newcount\colveccount
\newcommand*\colvec[1]{
        \global\colveccount#1
        \begin{pmatrix}
        \colvecnext
}
\def\colvecnext#1{
        #1
        \global\advance\colveccount-1
        \ifnum\colveccount>0
                \\
                \expandafter\colvecnext
        \else
                \end{pmatrix}
        \fi
}


\captionsetup{font={small}}

% Verlinktes Inhaltsverzeichnis
\usepackage[bookmarks,colorlinks=true,linkcolor=black]{hyperref}

\parskip 10pt
\setlength{\parindent}{0mm}
% Header
\pagestyle{scrheadings}
% loescht voreingestellte Stile
\clearscrheadings
\clearscrplain
\automark[section]{chapter}

\setheadsepline{1pt}        %Separate Linie im Kopf
\renewcommand*{\chapterheadstartvskip}{\vspace*{0pt}} % Abstand vor Kapitel�berschriften

\ihead[]{\leftmark}
\ohead[]{\rightmark}
\ofoot[\pagemark]{\pagemark}

\cfoot[]{}

\author{Jan-Frederik Schulte}
\title{CMS Dilepton edge search}

%%%%%%%%%%%%%%%%%%%%%%%%%%%%%%%%%%%%%%%%%%%%%%%%%%%%%%%%%%%%%%%%%%%%%%%%%%%%%%%%%%%%%%%%
%%%%%%%%%%%%     Titelseite/Inhalt      %%%%%%%%%%%%%%%%%%%%%%%%%%%%%%%%%%%%%%%%%%%%%%%%
%%%%%%%%%%%%%%%%%%%%%%%%%%%%%%%%%%%%%%%%%%%%%%%%%%%%%%%%%%%%%%%%%%%%%%%%%%%%%%%%%%%%%%%%
%\changefont{pbk}{b}{sc}
%Dokument
\begin{document}
\newcommand{\GeV}{\ensuremath{\,\mathrm{Ge\hspace{-.08em}V}}\xspace}
\newcommand{\GeVns}{\ensuremath{\mathrm{Ge\hspace{-.08em}V}}\xspace} % no leading thinspace
\newcommand{\gev}{\GeV}
\newcommand{\TeV}{\ensuremath{\,\mathrm{Te\hspace{-.08em}V}}\xspace}
\newcommand{\TeVns}{\ensuremath{\mathrm{Te\hspace{-.08em}V}}\xspace} % no leading thinspace

\newcommand{\lumi}{$19.5\,\fbinv$} %{1048~\pbinv}}
\newcommand{\fbinv}{\ensuremath{\mathrm{fb}^{-1}\xspace}} %{1048~\pbinv}}
\newcommand{\updated}{\textcolor{green}{This has been updated to the full luminosity}}
\newcommand{\notupdated}{\textcolor{red}{Not updated (possibly can't be updated ATM)}}
\newcommand{\significanceNumber}{$2.2\sigma$\xspace}
%\newcommand{\invpb}{\ensuremath{\textrm{pb-1}}}
\newcommand{\ttbar}{\ensuremath{t\bar{t}}\xspace}
\newcommand{\mc}{Monte Carlo\xspace}
\newcommand{\Red}{\color{red}}
\newcommand{\Z}{\ensuremath{\mathrm{Z}}\xspace}
\newcommand{\WZ}{\ensuremath{\mathrm{WZ}}\xspace}
\newcommand{\ZZ}{\ensuremath{\mathrm{ZZ}}\xspace}
\newcommand{\W}{\ensuremath{\mathrm{W}}\xspace}
\newcommand{\HH}{\ensuremath{\mathrm{H}}\xspace}
\newcommand{\Zmumu}{\ensuremath{Z\rightarrow \mu\mu}}
\newcommand{\Zee}{\ensuremath{Z\rightarrow ee}}
\newcommand{\Ztautau}{\ensuremath{Z\rightarrow \tau\tau}}
\newcommand{\Zll}{\ensuremath{Z\rightarrow ee,\mu\mu}}
\newcommand{\JZB}{\ensuremath{\textrm{JZB}}\xspace}
\newcommand{\jzb}{\JZB}
\newcommand{\mll}{\ensuremath{m_{\ell\ell}}\xspace}
\newcommand{\mlledge}{\ensuremath{m_{\ell\ell}^{edge}}\xspace}
\newcommand{\pt}{\ensuremath{p_{\mathrm{T}}}\xspace}
\newcommand{\Et}{\ensuremath{E_{\mathrm{T}}}\xspace}
% processes
%\newcommand{\Ztautau}{\ensuremath{\Z\to\tau\tau}}
\newcommand{\zjets}{\ensuremath{\Z+\text{jets}}\xspace}
\newcommand{\Zjets}{\zjets}
\newcommand{\Wjets}{\ensuremath{\W+\text{jets}}\xspace}
\newcommand{\DYjets}{\ensuremath{\text{DY}+\text{jets}}\xspace}
\newcommand{\gjets}{\ensuremath{\gamma+\text{jets}}\xspace}
\newcommand{\ttll}{$\ttbar\rightarrow\ell^+\ell^-$}
\newcommand{\tttau}{$\ttbar\rightarrow\tau^+\ell^-$}
\newcommand{\tthad}{$\ttbar\rightarrow{}q\bar{q}$}

\newcommand{\cls}{\ensuremath{\mathrm{CL_S}}}
\newcommand{\tauh}{\ensuremath{\tau_\text{h}}}
\newcommand{\effeet}{\ensuremath{\epsilon_{ee}^T}\xspace}
\newcommand{\effmmt}{\ensuremath{\epsilon_{\mu\mu}^T}\xspace}
\newcommand{\effemt}{\ensuremath{\epsilon_{e\mu}^T}\xspace}
\newcommand{\rmuestar}{\ensuremath{r_{\mu e}^{*}}\xspace}
\newcommand{\rmue}{\ensuremath{r_{\mu e}}\xspace}
\newcommand{\RT}{\ensuremath{R_\mathrm{T}}\xspace}
\newcommand{\nvert}{\ensuremath{N_{Vertices}}}
\newcommand{\njets}{\ensuremath{N_\text{Jets}}\xspace}
\newcommand{\mcut}{\ensuremath{m_\text{max}}}
\newcommand{\eps}{\ensuremath{\epsilon}}
\newcommand{\CLS}{\cls}
\newcommand{\Routin}{\ensuremath{R_\text{out/in}}\xspace}
\newcommand{\Rsfof}{\ensuremath{R_{\text{SF/OF}}}\xspace}
\newcommand{\Rmmof}{\ensuremath{R_\text{$\mu\mu$/OF}}\xspace}
\newcommand{\Reeof}{\ensuremath{R_\text{$ee$/OF}}\xspace}
\newcommand{\etalep}{\ensuremath{\eta_\text{lep}}\xspace}

\newcommand{\sSeven}{\ensuremath{\sqrt{s}=7\TeV}\xspace}
\newcommand{\sEight}{\ensuremath{\sqrt{s}=8\TeV}\xspace}
\newcommand{\lepSec}{The applied lepton selection changed in the \sEight-MC with respect to the \sSeven-MC}

\newcommand{\EM}{\ensuremath{e^\pm\mu^\mp}\xspace}
\newcommand{\EE}{\ensuremath{e^\pm e^\mp}\xspace}
\newcommand{\MM}{\ensuremath{\mu^\pm\mu^\mp}\xspace}

%PAS things
\newcommand{\Ht}{\HT}
\newcommand{\HT}{\ensuremath{H_\mathrm{T}}\xspace}
\newcommand{\Met}{\MET}
\newcommand{\MET}{\ensuremath{E_{\mathrm{T}}^{\mathrm{miss}}}\xspace}
\newcommand{\METVec}{\ensuremath{\vec{E}_{\mathrm{T}}^{\mathrm{miss}}}\xspace}
\newcommand{\ptll} {\ensuremath{\pt_{\ell\ell}}}
\newcommand{\CTEQ} {{\textsc{cteq}}}

% Miscellaneous macros
\newcommand{\forward}{forward\xspace}
\newcommand{\central}{central\xspace}
\newcommand{\inclusive}{inclusive\xspace}
\newcommand{\cfAN}[1]{(cf.~#1)}
\newcommand{\acAN}[1]{\cite{EdgeAN} Sec.~#1}
\newcommand{\acANApp}[1]{\cite{EdgeAN} App.~#1}
\newcommand{\benAN}[1]{\cite{TemplatesAN} Sec.~#1}
\newcommand{\fixme}[1]{{\color{red}\sffamily{\bfseries{}FIXME:} #1}}


\newcommand{\LowMetCentral}{\emph{LowMetCentral}}
\newcommand{\LowJetsCentral}{\emph{LowJetsCentral}}
\newcommand{\HighCentral}{\emph{HighCentral}}
\newcommand{\LowMetForward}{\emph{LowMetForward}}
\newcommand{\LowJetsForward}{\emph{LowJetsForward}}
\newcommand{\HighForward}{\emph{HighForward}}

%%%% FOR THE JZB SECTION %%%%%%%%%%%%%%%%%%%%%%%%%%%%%%%%%%%%%%%%%%%%%%%%%%%%%%%%%%%%%%%%%%%%%%%%%%%%%%
\definecolor{SFZP}{rgb}   {0.0 , 0.0 , 0.0} % signal region in black
\definecolor{OFZP}{rgb}   {0.6 , 0.0 , 0.0}
\definecolor{OFSB}{rgb}   {0.0 , 0.6 , 0.0}
\definecolor{SFSB}{rgb}   {0.0 , 0.0 , 0.6}
\definecolor{Bpred}{rgb}  {1.0 , 0.0 , 0.0} % average B pred in red 

\newcommand{\Bpred}{\ensuremath{{\color{Bpred}JZB^\mathrm{pred}_\mathrm{bkgd}}}}
\newcommand{\SFZPJZBPOS}{\ensuremath{{\color{SFZP} \textrm{JZB}^\mathrm{SF}_\mathrm{pos}}}} % signal region
\newcommand{\SFZPJZBNEG}{\ensuremath{{\color{SFZP} \textrm{JZB}^\mathrm{SF}_\mathrm{neg}}}} % JZB negative region
\newcommand{\OFZPJZBNEG}{\ensuremath{{\color{OFZP} \textrm{JZB}^\mathrm{OF}_\mathrm{neg}}}}
\newcommand{\OFZPJZBPOS}{\ensuremath{{\color{OFZP} \textrm{JZB}^\mathrm{OF}_\mathrm{pos}}}}
\newcommand{\SFSBJZBNEG}{\ensuremath{{\color{SFSB} \textrm{JZB}^\mathrm{SFSB}_\mathrm{neg}}}}
\newcommand{\SFSBJZBPOS}{\ensuremath{{\color{SFSB} \textrm{JZB}^\mathrm{SFSB}_\mathrm{pos}}}}
\newcommand{\OFSBJZBNEG}{\ensuremath{{\color{OFSB} \textrm{JZB}^\mathrm{OFSB}_\mathrm{neg}}}}
\newcommand{\OFSBJZBPOS}{\ensuremath{{\color{OFSB} \textrm{JZB}^\mathrm{OFSB}_\mathrm{pos}}}}

\newcommand{\secondchi}{\ensuremath{{\tilde{\chi}^0_2}}\xspace}
\newcommand{\firstchi}{\ensuremath{{\tilde{\chi}^0_1}}\xspace}
\newcommand{\sbottom}{\ensuremath{\tilde{b}}\xspace}
\newcommand{\slepton}{\ensuremath{\tilde{l}}\xspace}

%\newcommand{\SFZPJZchi2B}{\ensuremath{{\color{SFZP} \textrm{JZB}^\mathrm{SF}}}} % JZB negative region
%\newcommand{\OFZPJZB}{\ensuremath{{\color{OFZP} \textrm{JZB}^\mathrm{OF}}}}
%\newcommand{\SFSBJZB}{\ensuremath{{\color{SFSB} \textrm{JZB}^\mathrm{SFSB}}}}
%\newcommand{\OFSBJZB}{\ensuremath{{\color{OFSB} \textrm{JZB}^\mathrm{OFSB}}}}
%
\newcommand{\SFZP}{{\color{SFZP} SFZP}} % JZB negative region
\newcommand{\OFZP}{{\color{OFZP} OFZP}}
\newcommand{\SFSB}{{\color{SFSB} SFSB}}
\newcommand{\OFSB}{{\color{OFSB} OFSB}}

\newcommand{\TODO}[1]{\textcolor{red}{TO DO: #1}}

\newcommand{\PFSE}{\ensuremath{ \mathcal{P}_{FSE}}}

\newcommand{\eepm}{\Pep\Pem}
\newcommand{\mmpm}{\Pgmp\Pgmm}
\newcommand{\empm}{\ensuremath{\Pe^\pm \Pgm^\mp}}


\newcommand{\SFZPJZB}{\ensuremath{{\color{SFZP} \textrm{JZB}^\mathrm{SF}}}} % JZB negative region
\newcommand{\OFZPJZB}{\ensuremath{{\color{OFZP} \textrm{JZB}^\mathrm{OF}}}}
\newcommand{\SFSBJZB}{\ensuremath{{\color{SFSB} \textrm{JZB}^\mathrm{SFSB}}}}
\newcommand{\OFSBJZB}{\ensuremath{{\color{OFSB} \textrm{JZB}^\mathrm{OFSB}}}}




%Titelseite
%\thispagestyle{empty}
\begin{center}
  ~ \\
  \vspace{-0.5cm}
 {\huge\bf{Search for Supersymmetry in opposite-sign same-flavour dilepton events with the CMS detector}\\}
  \vspace{2.5cm}
    { Der Fakult\"at f\"ur Mathematik, Informatik und Naturwissenschaften der
RWTH Aachen University vorgelegte Dissertation zur Erlangung des akademischen Grades
eines Doktors der Naturwissenschaften\\}
  \vspace{2.5cm}
    {\Large von}\\
    {\Large\bf Jan-Frederik Schulte, M.Sc. RWTH}\\
    {\Large\bf aus M\"unster } \\
    
\end{center}
\cleardoublepage
\newpage
\newpage
%\pagenumbering{roman}
\section*{Liste der Vorver�ffentlichungen der Doktorabeit ``Search for Supersymmetry in opposite-sign same-flavour dilepton events with the CMS detector'' von Jan-Frederik Schulte.}
Ergebnisse dieser Doktorarbeit wurden von der CMS Kollaboration vorab in einem Artikel im Journal for High Energy Physics ver�ffentlicht: 
\begin{itemize}
\item CMS Collaboration, ``Search for physics beyond the standard model in events with two leptons, jets, and missing transverse momentum in pp collisions at $\sqrt{s}$ = 8 TeV'', $\textit{JHEP}$ $\textbf{1504}$ (2015) 124, doi:10.1007/JHEP04(2015)124, arXiv:1502.06031.
\end{itemize}






%Inhalt
%\tableofcontents



\end{document}