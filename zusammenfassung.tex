In dieser Arbeit wird eine Suche nach Supersymmetrie in Ereignissen mit zwei entgegengesetzt geladenen Leptonen gleichen Flavours, Jets und fehlender transversaler Energie pr\"asentiert. Der betrachtete Datensatz von Proton-Proton-Kollisionen bei einer Schwerpunktsenergie von 8\,TeV, aufgezeichnet mit den CMS-Detektor, entspricht einer integrierten Luminosit\"at von 19.5\,$\mathrm{fb}^{-1}$. Diese Analyse fokussiert sich auf die korrelierte Produktion von Elektron- oder Myonpaaren mit entgegengesetzter Ladung in Kaskadenzerf\"allen schwerer, supersymmetrischer Teilchen. Im Zerfall eines schweren Neutralinos in zwei Leptonen und ein leichteres Neutralino ergibt sich aus der Massendifferenz zwischen den Neutralinos eine obere Grenze auf die invariante Masse des Dileptonsystems $m_{\ell\ell}$. Dies f\"uhrt zu einer charakteristischen Kante in der $m_{\ell\ell}$-Verteilung. Weitere Teile der Signalsignatur umfassen mehrere hadronische jets und, da das leichteste supersymmetrische Teilchen, in diesem Fall das leichtere Neutralino, als stabil angenommen wird und den Detektor unbeobachtet verl\"asst, fehlende transversale Energie. Daher werden Ereignisse selektiert die wenigstens ein Paar entgegengesetzt geladener Leptonen gleichen Flavours enthalten und Anforderungen an die Anzahl von Jets und die fehlende transversale Energie erf\"ullen. Untergr\"unde aus bekannten Standardmodellprozessen, die zu dieser Selektion beitragen, sind entweder flavour-symmetrisch oder enthalten die korrelierte Produktion von Leptonen, etwa im Zerfall eines $\mathrm{Z}$-Bosons. Der dominierende flavour-symmetrische Untergrund kann mit hoher Genauigkeit aus Ereignissen mit entgegengesetzt geladenen Leptonen unterschiedlichen Flavours abgesch\"atzt werden. Kleine Korrekturen zur Flavour-Symmetrie dieser Prozesse, die von experimentellen Effekten hervorgerufen werden, werden mit zwei unabh\"angigen Methoden aus den Daten abgesch\"atzt. Beide Methoden liefern \"ubereinstimmende Ergebnisse. Die Untergr\"unde mit korrelierter Leptonproduktion sind ein nicht-dominanter Beitrag und werden auch aus den Daten abgesch\"atzt. Abweichungen der beobachteten Daten von diesen Erwartungen werden auf zwei Arten untersucht. In einem Z\"ahlexperiment wird die Anzahl beobachteter mit der Anzahl erwarteter Ereignisse verglichen. In einer Suche nach der charakteristischen Kantensignatur wird ein Fit an die $m_{\ell\ell}$-Verteilung durchgef\"uhrt. Dabei werden unterschiedliche Modelle f\"ur die zwei Untergrundarten und eine dreieckige Signalform verwendet. Der Fit findet eine Kantenposition von $82.4^{+2.1}_{-3.3}$\,GeV mit einer Signalst\"arke von $140\pm44$ Ereignissen f\"ur Leptonen im Zentralbereich des CMS-Detektors. Dieses Ergebnis entspricht einer lokalen Signifikanz von 2.5\,$\sigma$. Diese reduziert sich auf 1.7\,$\sigma$ wenn ber\"ucksichtigt wird, dass ein Signal an jedem Punkt der $m_{\ell\ell}$-Verteilung auftreten kann. Das Z\"ahlexperiment liefert konsistente Ergebnisse. Um den Einfluss dieser Ergebnisse auf Supersymmetrie zu studieren, werden die Ergebnisse des Z\"ahlexperiments in zwei vereinfachten Modellen von Paarproduktion von Bottom-Squarks interpretiert. 
\clearpage